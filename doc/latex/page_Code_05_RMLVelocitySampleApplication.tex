Below, you can find the source code of a sample application that makes use of the velocity-\/based Reflexxes algorithm and shows how to interpret the output values of the algorithm. The source file can be found in the folder~\newline
~\newline
{\ttfamily /src/\+R\+M\+L\+Velocity\+Sample\+Applications/05\+\_\+\+R\+M\+L\+Velocity\+Sample\+Application.cpp}~\newline
~\newline
\mbox{\hyperlink{page_Code_05_RMLVelocitySampleApplication_anc_VelocityExample5}{The resulting trajectory of this sample program is displayed below the source code.}} ~\newline
~\newline
 
\begin{DoxyCodeInclude}{0}
\end{DoxyCodeInclude}


~\newline
~\newline
\label{page_Code_05_RMLVelocitySampleApplication_anc_VelocityExample5}%
\Hypertarget{page_Code_05_RMLVelocitySampleApplication_anc_VelocityExample5}%
 

~\newline
 \begin{DoxyNote}{Note}
{\bfseries{The variables}}~\newline
~\newline

\begin{DoxyItemize}
\item \mbox{\hyperlink{classRMLInputParameters_a423bf4b1ef337cbf6eee22fe2e2502c1}{R\+M\+L\+Position\+Input\+Parameters\+::\+Current\+Acceleration\+Vector}} (containing the the current acceleration vector $ \vec{A}_i$ and~\newline
~\newline
 
\item \mbox{\hyperlink{classRMLInputParameters_a5968ce643868260410f149996c446b66}{R\+M\+L\+Position\+Input\+Parameters\+::\+Max\+Jerk\+Vector}} (containing the maximum jerk vector $ \vec{J}_i^{\,max} $ ~\newline
~\newline
 
\end{DoxyItemize}{\bfseries{are only used by the Type IV Reflexxes Motion Library.}}
\end{DoxyNote}
~\newline
~\newline
 \begin{DoxySeeAlso}{See also}
\mbox{\hyperlink{page_Result_05_RMLVelocitySampleApplication}{Screen Output of 05\+\_\+\+R\+M\+L\+Velocity\+Sample\+Application.\+cpp}} 

\mbox{\hyperlink{page_OutputValues}{Description of Output Values}} 

\mbox{\hyperlink{page_Code_02_RMLPositionSampleApplication}{Example 2 --- Making Use of Output Parameters of the Position-\/based algorithm}} 
\end{DoxySeeAlso}
